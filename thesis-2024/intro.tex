\section{Введение}
\subsection{Актуальность работы}

Кровь -- соединительная ткань внутри организма, она состоит из форменных клеток (эритроцитов, лейкоцитов, тромбоцитов), а так же из
водного раствора белков и свёртывающих веществ -- плазмы. Кровь под воздействием периодических сокращений сердечной мышцы 
движется по замкнутой системе сосудов, циркулируя от сердца и обратно. С точки зрения гидродинамики кровоток представляет 
из себя пульсирующее с низкой частотой течение мелкодисперсной суспензии в 
замкнутой системе каналов кругового сечения с эластичными стенками, осложнённое локальными эффектами ламинарно-турбулентного перехода.

Сложность разветвления кровеносных сосудов и вариации их размеров создают значительные трудности при решении задачи о течении крови. 
Математическое моделирование помогает  
аппроксимации и пониманию сложностей кровотока. Эти модели позволяют описывать и строить физические процессы, происходящие 
в биологических областях, что может быть полезно при выявлении, прогрессировании и лечении  различных сердечно-сосудистых заболеваний , 
а так же в проектировании и оптимизации медицинских устройств.

\subsection{Подходы к моделированию}

Описывать кровь можно различными способами. Например, детально: кровь состоит из взвешенных в плазме 
(которую чаще рассматривают, как ньютоновскую жидкость) клеток крови, которые действуют друг на друга с некоторыми силами. 
Описание такого типа методов можно подробнее изучить в ~\cite{Fedosov2010,Fedosov2008,Mehboudi2001}. 
В некоторых моделях ~\cite{bessonov:2014,hosseini:2009} пренебрегают относительно мелкими и редкими -- тромбоцитами
(2--4~мкм в количестве 150--300 миллионов на 1~см$^3$) и лейкоцитами(4--20~мкм в количестве 4.5 -- 11 миллионов на 1~см$^3$), 
а строят двумерные сетки, состоящие только лишь из эритроцитов (7 -- 8~мкм в количестве 3.8 до 5.6 миллиардов клеток на 1~см$^3$).
Но при таких подходах можно столкнуться с некоторыми проблемами: такую модель будет сложно сравнивать как
с другими моделями, так и с эксперементальными показателями, так же она очень плохо реагирует на любые изменения в исходных данных. 
Соответсвенно, возникает необходимость прибегнуть к дальнейшему осреднению.

Можно не описывать индивидуальные частицы взвешенные в плазме, а обобщить их до вязкой неньютоновской жидкости с определёнными 
характеристиками, тогда любые изменения можно будет отразить в параметрах жидкости. Такой подход называют трехмерным моделированием.
Существует множество трёхмерных моделей, например, основанные на зависимости вязкости от гематокрита ~\cite{walburn:1976},
модель Максвелла ~\cite{thurston:1972},  модель Кассона ~\cite{moller:2006}.
Однако эти модели всё-таки требуют значительных вычислительных ресурсов, а следовательно, и дальнейших упрощений. 

Для вывода граничных условий в трёхмерных моделях иногда используют одномерное моделирование, которое может быть 
и вполне самостоятельным подходом к моделированию течения крови.
В одномерных моделях пространственные характеристики осредняются по поперечному сечению, а трёхмерная дифференциальная
задача сводится к одномерной. Такой подход к моделированию требует меньших вычислительных ресурсов, но при этом почти не уступает в 
точности другим моделям.

