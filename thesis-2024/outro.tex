\section{Заключение}
В настоящей работе была рассмотрена нестационарная задача о течении крови в разветвленной сети сосудов с упругими стенками.
Кровь считалась вязкой ньютоновской жидкостью с постоянной плотностью.
Использовалась одномерная модель, в которой определяющие уравнения записываются относительно
средних по сечению сосуда скорости и давления, а также площади этого сечения.
Упругие свойства стенок сосуда описывались по равновесной модели в приближении однородной тонкой эластичной мембраны,
характеризуемой толщиной, модулем упругости и площадью поперечного сечения при отсутствии внешнего воздействия.
Дополнительным упрощением модели являлось использование постоянного радиально-симметричного профиля скорости.

Для определяющей системы уравнений был разработан вычислительный алгоритм. Для аппроксимации по пространству
использовался метод разрывных конечных элементов с Лагранжевыми элементами первого порядка.
Аппроксимация по времени осуществлялась с помощью схемы Кранка--Николсон.
Таким образом, аппроксимация и по времени и по пространству имели второй порядок точности.

Предложенная численная схема была верифицирована путём сравнения её результатов с известными
аналитическими и численными решениями модельных задач о течении крови в аналогичной одномерной постановке.
Рассматривались задачи о течении в одиночном однородном сосуде, о течении в сосуде с неупругой вставкой,
о течении в разветвлении сосудов.
Для всех рассмотренных задач получено удовлетворительное совпадение результатов.

В качестве иллюстрации применения алгоритма для расчёта более сложной системы
была рассмотрена задача течении в сети из семи сосудов,
один из которых имел повышенный модуль упругости и уменьшенный
радиус. Подобные изменения свойств характерны для склеротических поражений сосудов.
Было показано, что наличие таких повреждений в одном из сосудов
ведёт к значительному перераспределению
потоков на выходе из системы сосудов.
Причём разница в значении расходов на выходе из системы
растёт с увеличением частоты сердечных сокращений.
