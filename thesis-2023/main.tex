\documentclass[a4paper, 14pt]{article}

\usepackage[utf8]{inputenc}
\usepackage{graphicx}
\usepackage{float}%"Плавающие" картинки
\usepackage{wrapfig}%Обтекание фигур (таблиц, картинок и прочего)
\usepackage{amsmath, amssymb, amsthm}
\usepackage[T1, T2A]{fontenc}
\usepackage[english, russian]{babel}
\usepackage[left=2cm,right=2cm, top=2cm, bottom=2cm]{geometry}
% Литература в biblatex
\usepackage[backend=biber,bibencoding=utf8,language=auto,autolang=other,sorting=ntvy,babel=other,
            maxbibnames=99,maxcitenames=2,style=gost-numeric,movenames=false]{biblatex}
\addbibresource{../doc/refs.bib}
% Межстрочный интервал = 1.5pt
\usepackage{setspace}
\onehalfspacing
% % Абзацный отступ = 1.25см
\usepackage{indentfirst}
\setlength\parindent{12.5mm}
% Путь до папки с изображениями
\graphicspath{ {./images/} }
% Активные ссылки на формулы и лит-ру
\usepackage[
  colorlinks=true,
  citecolor=blue,
  linkcolor=blue,
  linktoc=page,
]{hyperref}
% tilde symbol
\usepackage{textcomp}
\renewcommand{\texttilde}{\raisebox{0.5ex}{\texttildelow}}
% Reynolds number
\newcommand{\Ren}{\mathrm{Re}}
\renewcommand{\vec}[1]{\boldsymbol{\rm #1}}

\begin{document}

\begin{titlepage}
\begin{center}

\hfill \break

\large{Министерство науки и высшего образования Российской~Федерации}\\
\footnotesize{Федеральное государственное автономное образовательное учреждение высшего образования}\\ 
\small{\textbf{«КАЗАНСКИЙ (ПРИВОЛЖСКИЙ) ФЕДЕРАЛЬНЫЙ УНИВЕРСИТЕТ»}}\\

\hfill \break
\normalsize{Институт механики и математики им. Н. И. Лобачевского}\\

\hfill \break
\normalsize{Направление подготовки: 01.01.03 Механика и математическое моделирование}\\

\vspace{25mm}
\large{Курсовая работа}\\
\large{\textbf{МЕТОДЫ МОДЕЛИРОВАНИЯ ТЕЧЕНИЯ КРОВИ}}\\
\end{center}

\vspace{20mm}
\noindent
Студентка 3 курса \\
группы 05-001 \\
<<\underline{\hspace{0,75cm}}>> \underline{\hspace{2cm}}\the\year~г. \hspace{9cm} Мымрина Е. В.

\hfill \break
Научный руководитель \\
доцент, к.н\\
<<\underline{\hspace{0,75cm}}>>\underline{\hspace{2cm}}\the\year~г. \hspace{9.1cm} Калинин Е. И.

\vspace{\fill}

\begin{center}
    Казань -- \the\year
\end{center}
\thispagestyle{empty}

\end{titlepage}


\newpage
\tableofcontents
\newpage

\section{Введение}
\subsection{Актуальность работы}

Кровь -- соединительная ткань внутри организма, она состоит из форменных клеток (эритроцитов, лейкоцитов, тромбоцитов), а так же из
водного раствора белков и свёртывающих веществ -- плазмы. Кровь под воздействием периодических сокращений сердечной мышцы 
движется по замкнутой системе сосудов, циркулируя от сердца и обратно. С точки зрения гидродинамики кровоток представляет 
из себя пульсирующее с низкой частотой течение мелкодисперсной суспензии в 
замкнутой системе каналов кругового сечения с эластичными стенками, осложнённое локальными эффектами ламинарно-турбулентного перехода.

Кровь, двигаясь по сосудам, испытывает сопротивление движению со стороны сосудов и из-за своей вязкости. Поэтому сердце вбрасывает 
кровь в сосуды под большим давлением. В аорте давление колеблется в диапазоне от 16~кПа при систоле до 10~кПа при диастоле. 
По мере движения крови давление в сосудистом русле падает. 
Скорость течения крови так же зависит от диаметра сосуда, удалённости сосуда от сердца, а также фазы сердечного цикла. 
Максимальных значений скорость достигает в аорте (до \texttilde$1$~м/с), а минимальных -- в капиллярах (около нуля).

Сложность разветвления кровеносных сосудов и вариации их размеров создают значительные трудности при решении задачи о течении крови. 
Математическое моделирование помогает  
пониманию сложностей кровотока. Эти модели позволяют описывать и строить физические процессы, происходящие 
в биологических системах. Это может быть полезно при выявлении, прогрессировании и лечении  различных сердечно-сосудистых заболеваний , 
а так же в проектировании и оптимизации медицинских устройств.

Кровь состоит из взвешенных в плазме 
(её рассматривают, как ньютоновскую жидкость) клеток крови, которые действуют друг на друга с некоторыми силами. 
Самые детальные методы моделирования заключаются в построении модели течения этих клеток как отдельных частиц в вязкой жидкости.
Описание такого типа методов можно подробнее изучить в ~\cite{Fedosov:2010,Fedosov:2008,Mehboudi:2001}. 
В некоторых моделях ~\cite{bessonov:2014,hosseini:2009} пренебрегают относительно мелкими и редкими -- тромбоцитами
(2--4~мкм в количестве 150--300 миллионов на 1~см$^3$) и лейкоцитами(4--20~мкм в количестве 4.5 -- 11 миллионов на 1~см$^3$), 
а моделируют лишь самые крупные из них -- эритроциты (7 -- 8~мкм в количестве 3.8 до 5.6 миллиардов клеток на 1~см$^3$).
Метод ресурсоёмкий, поэтому можно моделировать лишь небольшие участки кровотока (порядка 1~см и меньше) без ветвлений 
либо с одним - двумя ветвлениями.
Так же такая модель плохо реагирует на любые изменения в исходных данных, ведь свойства эритроцитов могут значительно изменяться, 
а модель строится под определённую их конфигурацию~\cite{Yamaguchi2010}. 
Соответсвенно, возникает необходимость прибегнуть к некоторым упрощениям постановки.

Можно не описывать индивидуальные частицы взвешенные в плазме, а обобщить их до вязкой неньютоновской жидкости с определёнными 
характеристиками, в которых отразить эффекты, связанные с наличием взвешенных в растворе частиц. Такой подход называют трехмерным моделированием.
Принципиальным моментом в формулировке такой постановки является выбор модели вязкости: например,
основанные на зависимости вязкости от гематокрита ~\cite{walburn:1976},
модель Максвелла \cite{thurston:1972},  модель Кассона ~\cite{moller:2006}.
Однако эти модели всё-таки требуют значительных вычислительных ресурсов, поэтому имеет смысл провести дальнейшее упрощение.

Наиболее простыми с точки зрения вычислительных ресурсов являются так называемые одномерные модели кровотока, в которых
пространственные характеристики осредняются по поперечному сечению, а трёхмерная дифференциальная
задача сводится к одномерной.
В некоторых случаях эти модели используют для постановки граничных условий в многомерных задачах.
Но так же они могут и целиком моделировать кровеносную систему.
Такой подход к моделированию требует меньших вычислительных ресурсов, но при этом почти не уступает в 
точности другим моделям. О сравнении одномерных и многомерных моделей можно прочитать в ~\cite{FORMAGGIA:2001}.

{\bfВывод об актуальности}
Таким образом, несмотря на значительное увеличение доступных вычислительных мощностей и активное развитие детальных методов
численного моделирования гемодинамики, до сих пор полное описание системы кровообращения подробными моделями остаётся невозможным, 
а прямое или косвенное использование одномерных моделей для таких задач является безальтернативным выбором.
Поэтому развитие таких упрощённых моделей является актуальной задачей.

\subsection{Обзор методов моделирования на основе одномерной постановки}

Принципиальным вопросом в построении одномерной модели кровотока является выбор зависимости давления в сосуде от площади его
поперечного сечения.

Можно использовать $N$ интегральных условий сохранения Бернулли выражающих непрерывность полного давления $P$:
\begin{equation*}
    \label{eq:bernulli}
    \frac{\rho u^2_k}{2}+{p_k(A_k)}=P.
\end{equation*}
Иногда для моделирования сопротивления потоку в местах стыка используют условия перепада давлений
с учётом сопротивления
\begin{equation*}
    \label{eq:p-pressure}
    p_k\left(A_k\left(t,x_k\right)\right)-p^l(t)=\varepsilon_k R^l_k A_k(t,x_k)u_k(t,x_k),
\end{equation*}
которое выражается через коэффициент $R^l_k$. Здесь $p^l$ -- давление в точке бифуркации \cite{bessonov:2014}.

Но общим способом замыкания системы является явное представление алгебраической зависимости
между давлением в сосуде и его площадью. Прямой подход к получению отношения $p(A)$ включает в себя точное одновременное измерение
давления и площади в разные моменты времени. Но такой метод не всегда удобен в реальности.
Качественный анализ физических экспериментов подтверждает, что функция $p(A)$ должна быть монотонной S-подобной кривой. 
Такая кривая удовлетворительно описывает состояния как круглого, так и эллиптического сечения. На рис.~\ref{ych} показана зависимость
зависимость давления в сосуде от его площади.

\begin{figure}[h]
    \centering
    \includegraphics[width=0.5\linewidth]{PA.png}
    \caption{График зависимости давления от поперечного сечения внутри сосуда}
    \label{ych}
\end{figure}

Для решения дифференциальных уравнений нужно выбрать подходящий метод пространственной аппроксимации области расчёта.
Ниже привёдем обзор наиболее часто используемых методов аппроксимации.

{\bf Метод конечных разностей} предполагает дискретизацию области на сетку точек и последующую аппроксимацию производных в управляющих
уравнениях с помощью конечных разностей. Этот метод часто используется в сочетании со схемой интегрирования по времени для решения 
полученной системы обыкновенных дифференциальных уравнений.

{\bf Метод конечных элементов}~\cite{TAYLOR1998} -- популярная численная схема. 
Он особенно хорошо подходит для моделирования сложных геометрий, таких как запутанная сеть артерий в человеческом теле. 
Однако он может быть вычислительно дорогим, особенно для больших и сильно разветвленных сетей. 

{\bf Метод быстрого преобразования Фурье}~\cite{Sazonov:2019} -- 
это подход, использующий быстрое преобразование Фурье для решения одномерных уравнений кровотока. 
Этот метод конкурирует с традиционными пространственно-временными численными схемами как по устойчивости, так и по скорости. 
Он может точно и эффективно обрабатывать сложные геометрические формы и высокоамплитудные волны. 
Однако он требует дальнейшего развития для учета вязкоупругих эффектов и потери массы крови из-за мелких ветвей. 

{\bf Метод разрывных конечных элементов Галеркина}~\cite{yao:2017} -- это еще одна численная схема, она сочетает в себе преимущества методов конечной разности и конечных элементов, 
обеспечивая баланс между точностью и вычислительными затратами. Однако он может быть более сложным в реализации и может
потребовать дополнительных вычислительных ресурсов для сопоставления расчетной и физической областей.

{\bf Метод конечных объемов с локальным временным шагом высокого порядка}~\cite{mueller:2015} предполагает решение управляющих уравнений 
с помощью метода конечных объемов высокого порядка и схемы локального шага по времени. 
Этот метод может быть особенно полезен для моделирования течения в сложных геометрических системах.

\subsection{Цель работы}
Целью настоящей работы является:
\begin{itemize}
\item разработка методики расчёта течения крови в одномерном приближении,
\item написание компьютерной программы для расчёта произвольной сети сосудов,
\item подбор оптимальных расчётных параметров: шагов по времени, пространству, количеству внутренних итераций,
\item верификация расчётной программы путём сравнения с численными и аналитическими результатами расчётов задачи в характерных модельных постановках от других авторов,
\item иллюстрация работы программы на примере сложной сети сосудов. Влияние изменения эластичных свойств одного из сосудов на интегральные характеристики течения 
      во всей системе.
\end{itemize}

\section{Подходы к моделированию кровотока}
\subsection{Детальное моделирование}
Моделировать кровь можно как жидкость со взвешенными в ней клетками крови, которые могут рассматриваться, как совокупность частиц и сил,
действующих между ними. Эритроциты гораздо крупнее остальных клеток крови и они составляют большую часть её объёма, соответственно они 
и будут определять механические свойства крови. Плазма крови -- раствор крупных молекул, но при масштабах движения и при скоростях
сдвига, обычно встречающихся в кровеносных сосудах, ее можно считать однородной ньютоновской жидкостью  и описывать уравнениями
Новье-Стокса. То есть нам нужно учитывать взаимодействие частиц с жидкостью и друг с другом. Такую задачу удобно решать методом 
диссипативной динамики частиц.

Диссипативная динамика частиц -- это метод, в котором каждая частица описывает небольшой объем моделируемой среды, 
а не отдельную молекулу. Их взаимодействие определяется консервативными $F^C_{ij}$ (не зависящими от траектории), диссипативными 
$F^D_{ij}$ (силы, при действии которых полная механическая энергия  системы убывает, переходя в другие, не механические формы энергии) 
и случайными силами $F^R_{ij}$, действующими между двумя частицами:
\begin{align*}
  &\vec{F^C_{ij}}=F^C_{ij}(r_{ij})\vec{\hat{r}{_{ij}}},\\[10pt]
  &\vec{F^D_{ij}}=-\gamma \omega^D(r_{ij}) (\vec{v_{ij}} \cdot \vec{\hat{r}{_{ij}}})\vec{\hat{r}{_{ij}}},\\[10pt]
  &\vec{F^R_{ij}}=\sigma \omega^R(r_{ij}) \dfrac{\varepsilon_{ij}}{\sqrt{\bigtriangleup t}} \vec{\hat{r}{_{ij}}},
\end{align*}
где $\vec{r_{i}}$— радиус-вектор i-ой частицы, $\vec{r_{ij}}=\vec{r_{j}} - \vec{r_{j}}$,
$r_{ij}=|\vec{r_{ij}}|$,
$\vec{\hat{r}{_{ij}}}=\vec{r_{ij}}/{r_{ij}}$,
$\vec{v_{ij}}=\vec{v_{j} - \vec{v_{j}}}$ -- разница между скоростями двух частиц, $\bigtriangleup t$ -- шаг по времени, 
$\gamma, \sigma$ -- это  коэффициенты, определяющие силу диссипативной и случайной силы соответственно, а $\omega^R,\omega^D$ -- весовые функции,
${\varepsilon_{ij}={\varepsilon_{ji}}}$ -- нормально распределенная случайная величина с нулевым средним и единичной дисперсией.

Мембрана эритроцита практически несжимаема и устойчива к изменению площади поверхности деформации сдвига в плоскости. 
Она может быть смоделирована как двумерная сеть частиц соединенных пружинами, смоделированными по закону Гука 
и образующих неправильный многогранник с треугольными гранями Рис.~\ref{elasticity scheme}.

\begin{figure}[h]
\centering
\includegraphics[width=0.2\linewidth]{mol3.png}
\caption{ Схема упругости при растяжении и изгибе между частицами в мембране~\cite{hosseini:2009}.}
\label{elasticity scheme}
\end{figure}

Для её описания используется четыре силы.
Первая сила возникает, когда стороны треугольников меняют свою длину.
$$
\vec {F_I}=k_I\left(1- \frac{l}{l_0}\right)\vec\tau,
$$
где $l$ -- длина ребра, $l_0$ -- равновесная длина, $k_I$ -- коэффициент жёсткости, $\vec {\tau}$ -- единичный вектор, 
направленный от одной вершины к другой.

К каждому узлу сходится несколько сторон, и каждая из них вносит свой вклад в общую силу F. 
Вторая сила определяется сжатием или расширением мембраны 
$$
{p_s}=k_s\left(1- \frac{s}{s_0}\right),
$$
где $s$ -- площадь треугольника, $s_0$ -- площадь равновесного треугольника, $k_s$ -- коэффициент расширения площади.

Третья сила пропорциональна изменению объема многогранника и прикладывается к вершинам треугольников в направлении, нормальном к поверхности 
$$
\vec{F_v}=k_v\left(1- \frac{v}{v_0}\right) s \vec{n},
$$
где $v$ -- объем многогранника, $v_0$ -- равновесный объем, $k_v$ -- коэффициент, $s$ -- площадь
треугольного элемента, а $n$ -- единичный нормальный вектор к этому треугольнику.

Четвёртая сила выражается через изгибающий момент~\cite{hosseini:2009} 
$$
 \vec{M_i}=k_M \tan\left(\frac{\theta}{2}\right)l \vec{\tau_i},
$$
где $\theta$ -- угол между соседними треугольными элементами, 
$k_m$ -- коэффициент жесткости, $\vec{\tau_i}$ -- единичный вектор, сонаправленный с общим краем двух треугольников 
и $l$ -- длина этой стороны.

Аналогичным образом могут быть смоделированы и другие клетки крови. Например, в отличие от эритроцитов, лейкоциты содержат ядро. 
Из-за этого они менее деформируемы, а их форма близка к сфере. Это влияет на их поведение в потоке крови.

Для определения параметров эритроцитов прибегают к методам численного моделирования~\cite{bessonov:2014}.

Фактически, можно легко указать на несколько недостатков данной модели. Например, рассмотрение всей клетки как однородной решетки 
делает невозможным учет любой органеллы в цитоплазме. Кроме того, эффективная упругость решетки зависит не только от 
пружинной постоянной, но и от конкретной формы, размера и топологии решетки. Это затрудняет сравнение между различными моделями 
и экспериментами. Более того, параметры модели на уровне частиц  модели не были систематически связаны с физиологическими измерениями; 
они должны принимать нереальные значения, чтобы предсказание было количественно сопоставимо с реальностью 


\subsection{Моделирование крови как вязкой жидкости}
Жидкость может считаться ньютоновской, если она удовлетворяет закону вязкости Ньютона, 
т.е. напряжение сдвига пропорционально скорости сдвига, а вязкость является константой пропорциональности, 
поэтому плазму крови, состоящую в основном из воды, можно считать ньютоновской. Однако у крови более сложные механические свойства. 
И мы предполагаем, что все макроскопические масштабы длины и времени достаточно велики по сравнению с масштабами длины и времени 
на уровне отдельного эритроцита. Таким образом, для упрощения построения модели и дальнейшей работы с ней можно рассматривать 
взвешенные клетки в плазме, как вязкую несжимаемую, неньютоновскую жидкость.

Такие методы применяется в тех сосудах, где клетки достаточно малы и многочисленны, чтобы их отдельную индивидуальность можно было
игнорировать, а их влияние на движение всей крови описывать усредненным способом. 
Так обстоит дело в крупных артериях (диаметр аорты, например, примерно в 2000 раз больше диаметра эритроцита). 
На практике эти методы применяются в изучении и моделировании сердечного выборса, печеночного кровотока, среднего артериального давления.

\textbf{Вязкость.}
Для начала рассмотрим простейшую конститутивную модель, основанную на предположении, что тензор дополнительных напряжений пропорционален
симметричной части градиента скорости.
$$
\tau=2\mu (\dot{\gamma})D,
$$
где $\tau $ -- тензор дополнительных напряжений, $D=(\nabla u+\nabla u^T)$ -- тензор скорости деформации, 
$\mu (\dot{\gamma})$ -- скорость сдвига.
Зависимость $\mu (\dot{\gamma})$ находят разными способами. Например, в~\cite{walburn:1976}, 
рассматривают зависимость вязкости от гематокрита и белка минус-альбумин: \\
$$
\mu (\dot{\gamma})=K{\gamma}^{n-1}, \qquad  K=C_1 \exp (C_2 H_t).
$$
Полученные таким способом результаты вязкости оказались схожи с результатами, которые были получены в~\cite{kim:2000} при помощи вязкозиметра.\\

\textbf{Вязкоупругость.}
Вязкоупругие жидкости -- это вязкие жидкости, обладающие способностью накапливать и высвобождать энергию. 
Упругая энергия объясняется свойствами мембраны РНК, демонстрирующей релаксацию~\cite{evans:1976} и зависящей от скорости сдвига.\\
Одной из простейших моделей, учитывающих вязкоупругость крови, является модель Максвелла~\cite{thurston:1972}:
$$
\tau+\lambda_1(\delta\tau / \delta t)=2\mu D,
$$  
где $\lambda_1$ -- время релаксации, a $\delta\tau / \delta t$ -- обобщение материальной производной по времени.
При построении данной модели полагают, что время релаксации зависит от скорости сдвига.
В~\cite{thurston:1994} данный метод используют для понимания неньютоновского вязкого характера крови после устойоявшегося поток.\\

\textbf{Предел текучести.}
Предел текучести обычно рассматривается как постоянное материальное свойство жидкости. 
Но зафиксированные пределы текучести имели большой разброс, который, стоит отметить, зависел от множества факторов.
Поэтому предел текучести стоит рассматривать не как константу, а как функцию времени и связанную с тиксотропией, 
как было предложено в~\cite{moller:2006}.
Более популярной моделью предела текучести является модель Кассона.
$$
\begin{aligned}
	&\sqrt{II_\tau} < \tau_Y\longrightarrow D=0 \\
	&\sqrt{II_\tau} \geq \tau_Y\longrightarrow
	\begin{cases}
		D   = \dfrac{1}{2\mu N}\left(1-\dfrac{\sqrt{\tau_Y}}{II_\tau}\right)^2\tau \\[10pt]
		\tau= 2\left(\sqrt{\mu N}+\dfrac{\sqrt{\tau_Y}}{\sqrt[4]{4II_D}}\right)^2
	\end{cases}
\end{aligned}
$$

В нашем случае предел текучести является параметра крови, например, для понимания, как будет вести себя кровь под влиянием какого-либо 
препарата, не станет ли она слишком вязкой или наоборот.
Однако, до сих пор использование предела текучести в качестве параметра моделирования крови остаётся спорным, 
в связи с неточностями его определения.

\subsection{Одномерные модели}
Данное семейство методов основано на осреднении трёхмерных характеристик течения по поперечному сечению и сведении
трёхмерной дифференциальной задади о движении крови по сосуду к одномерной.
Для описании сети кровеносных сосудов используются условия сохранения расхода и полного давления в точках бифуркации.

Привлекательность oдномерныx моделeй кровотока основана на разумных вычислительных затратах. 
Основные приложениями этих моделей являются транспорт газов крови и лекарств в организме, перераспределение кровотока при внешних 
или внутренних воздействиях, перераспределение кровотока в результате внутрисосудистых операций. Основными недостатками данных моделей 
являются сложность численных схем, высокая стоимость вычислений и спекулятивные граничные условия. 
Вычислительная область -- это сосудистая одномерная сеть человека или ее части. 
Сеть может быть создана на основе общих анатомических данных, таких как справочник анатомических карт~\cite{bunicheva:2013}, 
анатомические 3D модели или данные конкретного пациента.

Обозначим через $t$ -- время, через $x$ -- координату вдоль сосуда, через $A(t, x)$ -- площадь поперечного сечения сосуда, 
$u(t, x)$ и $p$ -- усредненную по сечению линейную скорость и трансмуральное давление крови. Баланс массы и импульса в переменных 
$(A, u, p)$ соответствует уравнениям
\begin{align}
    \label{eq:mass-balance}
    \varphi=&\frac{\partial A}{\partial t}+\frac{\partial Au}{\partial x},\\
    \label{eq:momentum-balance}
    \psi=&\frac{\partial u}{\partial t}+ \frac{\partial u^2/2+p/\rho}{\partial x}
\end{align}
Правые части уравнений (масса источника или поглотителя на единицу длины, ускорение в результате внешних воздействий) зависят от моделируемого процесса. 

Неизвестными в этих уравнениях являются осреднённые значения $u$, $p$ и $A$.
В одномерных моделях кровотока традиционное описание упругих свойств стенки сосуда обеспечивается зависимостью давления от площади 
поперечного сечения $p(A)$. Прямой подход к получению отношения $p(A)$ включает в себя точное одновременное измерение в естественных 
условиях давления и площади в разные моменты времени. Но такой метод не всегда удобен в реальности.

Качественный анализ физических экспериментов подтверждает, что функция $p(A)$ должна быть монотонной S-подобной кривой. 
Такая кривая удовлетворительно описывает  состояния как круглого, так и эллиптического сечения. 
На практике S-подобная зависимость давления от площади часто аппроксимируется аналитической функцией. 

\begin{figure}[h]
\centering
\includegraphics[width=0.3\linewidth]{IMG_20230309_021324_943-01.jpeg}
\caption{ График зависимости $p(A)$~\cite{pedly:1998}.}
\label{fig:mpr}
\end{figure}
Другие обобщения модели эластичной стенки сосуда являются модели вязкоупругой стенки. В таких моделях зависимость выражается в виде
$
p(A)=F(A,\partial {A} / \partial {t},\partial^2{A} / \partial {t^2},\partial^2{A} / \partial {x^2})
$.

%Эта модель была использована в~\cite{bunicheva:2004} для моделирования воздействия гравитационной перегрузки на системное кровообращение. 

Правильные упругие свойства артерий и вен могут быть описаны следующей функцией~\cite{holodov:2001}:
\begin{equation}
    \label{eq:elastic-propeties}
    p(A)=\rho c^2_0 f(a/a_0), 
    \quad
    f(k)=\begin{cases}
    \exp(k-1)-1, &k>1 \\ \ln(k), &k \leq 1
    \end{cases}
\end{equation}

В случае рассмотрения сети сосудов уравнения сохранения массы и импульса(\ref{eq:mass-conserv}),(\ref{eq:mom-conserv}). 
записываются для каждого k-го сосуда. C учётом упругих свойств сосудов (\ref{eq:elastic-propeties}) получим  
\begin{align}
    \label{eq:mass-conserv}
    \frac{\partial A_k}{ \partial t} + \frac{\partial(u_kA_k)}{\partial x}&=\varphi _k(t,x,S_k,u_k,r_i),\\
    \label{eq:mom-conserv}
    \frac{\partial u_k}{\partial t} + \frac{\partial(u_k^2/2+p_k/\rho_k)}{\partial x}&= \psi_k(t,x,S_k,u_k,r_i),\\
    \label{eq:t-pressure}
    p_k(t,x)-p_*(t,x)&=\rho_k c^2_{k0}f_k(S_k(t,x)),
\end{align}
где $t$ -- время, $x$ -- расстояние вдоль сосуда, $\rho$=const -- плотность, $c_{k0}(t,x)$ -- скорость распространения малых возмущений,
  $p_k(t,x)$ -- давление внутри сосуда (от атмосферного) $p_*(t,x)$ -- избыточное давление в тканях, окружающих сосуд, 
  $A_ku_k=Q(t,x)$ -- объёмный расход, $\varphi_k(t,x,S_k,u_k,r_i)$ -- источник/утечка массы, $\psi_k(t,x,S_k,u_k,r_i)$ -- внечние силы (гравитация, трение и т.~д.),
  $k=1,2,\ldots$ -- индекс сосуда.
\\ 

{\bf Граничные условия.}
Для одномерных моделей кровотока необходимы граничные условия на стыках сосудов, входах и выходах в сети сосудов.
Для всех сосудов граничные условия должны включать условия совместимости по характеристикам системы гиперболических уравнений (\ref{eq:mass-balance}), (\ref{eq:momentum-balance})~\cite{Xiu:2007}.
В каждой конечной точке сосуда требуется только одно дополнительное условие совместимости.
Вторым условным граничным условием на стыке N сосудов является сохранение массы:
\begin{equation}
    \label{eq:conserv-mass}
    \sum_{k=k_1,k_2,...,k_N} \varepsilon_k A_k(t,x_k)u_k(t,x_k)=0,
\end{equation}
где {$k_1,...,k_N$} -- индексы сосудов, $\varepsilon_k=1, x_k=0$ -- для входящих сосудов,
$\varepsilon_k=1, x_k=L_k$ -- для выходящих сосудов.
Также используется $N$ интегральных условий сохранения Бернулли выражающих непрерывность полного давления $P^l$:
\begin{equation}
    \label{eq:bernulli}
    \frac{\rho u^2_k}{2}+{p_k(A_k)}=P^l.
\end{equation}
Иногда для моделирования сопротивления потоку в местах стыка используют условия перепада давлений
с учётом сопротивления
\begin{equation}
    \label{eq:p-pressure}
    p_k\left(A_k\left(t,x_k\right)\right)-p^l(t)=\varepsilon_k R^l_k A_k(t,x_k)u_k(t,x_k),
\end{equation}
которое выражается через коэффициент $R^l_k$. Здесь $p^l$ -- давление в точке бифуркации \cite{bessonov:2014}.


Выходы сети артерий и входы сети вен должны быть связаны с множеством мелких неучтенных сосудов, относящихся к микрососудистым регионам.
 Поток в таких сосудах не может быть описан одномерными моделями течения (\ref{eq:mass-balance})-(\ref{eq:mass-conserv})
 из-за большого количества сосудов, сложной структуры микрососудистых сетей и неньютоновской реологии крови.

Более простой подход заключается в том, чтобы объединить артерий и вены в точки соединения,
где выполняются (\ref{eq:conserv-mass}),(\ref{eq:p-pressure}). Коэффициенты сопротивления $ R^l_k$ оцениваются  по известному перепаду
давления между артериями и венами. Чем больше сосудистая сеть, тем меньше сосудов объединяются, тем выше точность метода.
Более подходящие граничные условия оттока: отдельные участки мелких артерий и микроциркуляции моделируются как структурированные деревья,
чей импеданс корней может быть оценен из линеаризации управляющих уравнений.

Многие авторы~\cite{alastruey:2008} выполняют сопряжение во временной области 1D моделей кровотока с электрическими цепями
(единичными параметрами) 0D моделей. 0D модели обеспечивают корректные граничные условия для глобальных моделей кровотока.
\\
\subsection{Вывод}
Были рассмотрены три семейства методов моделирования течения крови в организме.
Методы детального моделирования рассматривают течение мелкодисперсной суспензии
и подробно учитывают эффекты связанные с влияние клеток крови на поток.
Из-за большой вычислительной сложности эти методы можно применять
только для расчётов очень небольших участков сосудов.
Методы моделирования, основанные на решении задачи о течении вязкой жидкости,
пренебрегают прямым учётом наличия клеток в потоке, а связанные с этим эффекты
учитываются через модификацию значения вязкости потока.
Такие методы менее требовательны к вычислительной мощности и имеют более широкое применение. С их помощью могут быть решены
задачи о течении в отдельном сосуде или даже нескольких связанных сосудах.
Методы одномерного моделирования решают задачу в одномерной осреднённой постановке и могут быть
применены для расчёты всей сердечно-сосудистой системы в целом.

\section{Примеры расчётов кровотока}
\subsection{Расчёт полной системы циркуляции}
Для примера применения одномерной модели кровотока для расчёта полной системы циркуляции рассмотрим
задачу о влиянии атеросклеротической бляшки на параметры течения крови, изученной в работе \cite{vassilevski:2011}.

На первом этапе решения задачи строится граф, описывающий изучаемую циркуляционную систему.
Он представленный двумя связанными сетями артерий и вен (см. Рис.~\ref{ss}). 
Сосудистая система состоит из 341 сосуда с анатомически адекватными свойствами (длина, диаметр, упругие свойства), вен и артерий. 
Вены и артерии соединены в 162 точках, на которые наложены граничные условия (\ref{eq:conserv-mass}),(\ref{eq:p-pressure}). 

\begin{figure}[h]
\centering
\includegraphics[width=0.5\linewidth]{krug.png}
\caption{Упрощённая структура сосудов системного круга. А—артерии, Б—вены.}
\label{ss}
\end{figure}

Далее в каждом сосуде вводится одномерную равномерную сетку и дискретизируем систему (\ref{eq:mass-balance}),(\ref{eq:momentum-balance}) 
методом монотонных характеристик первого порядка. Уравнения расширяются набором жестких ОДЕ, которые описывают работу сердца в терминах 
усредненной по объему модели.

Система жестких ОДУ, решаемая неявным методом Рунге-Кутта третьего порядка, обеспечивает граничные условия на входе и выходе сердца. 
Алгебраическая дифференциальная система (\ref{eq:mass-balance}),(\ref{eq:momentum-balance}), (\ref{eq:conserv-mass}),(\ref{eq:p-pressure}) 
в сочетании с зависимостью давления от площади и соответствующими граничными условиями на входе и выходе сердца 
и микроциркуляторных областях, решается по схеме с дробным шагом по времени схема, которая разделяет вычисления на 
локальные независимые части (отдельные сосуды и отдельные точки соединения).

На гиперболическом подэтапе применяется явный метод характеристик для каждого сосуда и контролируем шаг по времени с помощью 
ограничения устойчивости $\tau = 0.9 s_{\max}$, $s_{\max}=\max_{k,i}|\lambda _{k,i}|/h_k$, где $h_k$ -- размер сетки в сосуде $k$, 
$\lambda$ -- размер сетки в сосуде $k$,  $\lambda _{k,i}$ -- наибольшее (по величине) собственное значение якобиана для  (\ref{eq:mass-balance}),(\ref{eq:momentum-balance}), 
 в точке сетки.

В алгебраическом подэтапе применяется метод Ньютона для системы уравнений в каждом узле пересечения. 
Система состоит из уравнений (\ref{eq:conserv-mass}),(\ref{eq:p-pressure}) и условие совместимости по характеристикам (\ref{eq:mass-balance}),(\ref{eq:momentum-balance}). 
Влияние атеросклеротической бляшки учитывается в модели эластичной стенки. Здоровые сосуды описываются уравнением (\ref{eq:elastic-propeties}), 
обеспечивающим достоверную корреляцию с экспериментальными кривыми. Атеросклеротические артерии рассматриваются как 
трехслойные цилиндрические оболочки, деформированные внутренним давлением крови. Внутренний и внешний слои оболочки -- это 
фиброзная крышка и стенка артерии, соответственно. Деформации фиброзной пробки и стенки сосуда моделируются с помощью 
волоконно-эластичной модели. В простейшей версии волоконной осесимметричной модели оболочка представлена набором кольцевых волокон, 
которые сопротивляются только растяжению и сжатию волокон, как неогуковские материалы.

\begin{equation}
    \label{loc-force}
    \vec{F}=\frac{\partial}{\partial s}(T\vec{\tau}),
    \quad
    T=\mu\left(\left|\frac{\partial \vec{X}}{\partial s}\right|^2-\left|\frac{\partial \Vec{X}}{\partial s}\right|^{-2}\right).
\end{equation}
Здесь $\mathbf{F}$ обозначает плотность локальной силы, $T(s)$ обозначает натяжение волокна, 
$\Vec{\tau} =\partial \Vec{X}/\partial s^{-1}$ единичный касательный вектор, $\Vec{X}(s)$ 
представляет собой положение точек волокна в пространстве, координата Лагранжа s -- длина дуги волокна в ненапряженном состоянии. 
Липидный пул (промежуточная оболочка) имитируется набором радиальных пружин с нелинейной зависимостью между силой реакции и смещением

\begin{figure}[h]
\centering
\includegraphics[width=0.4\linewidth]{chast.png}
\caption{Участок сосудов системного круга.}
\label{ych}
\end{figure}

\begin{figure}[h!]
\centering
\includegraphics[width=0.45\linewidth]{94.jpg}
\includegraphics[width=0.45\linewidth]{102.jpg}\\
\includegraphics[width=0.45\linewidth]{104.jpg}
\caption{Скорость (см/с) в сосудах 94, 102, 104 для различных просветов. Здоровый сосуд
соответствует 100\% просвету}
\label{sc}
\end{figure}

Отношение давления к площади атеросклеротической артерии получено из предположения о статическом равновесии стенки: 
внутреннее давление крови уравновешивается упругими силами вышеупомянутой системы волоконных пружин, возникающими при ее смещении. 
На основе смещений можно рассчитать поперечную площадь сечения $A$ как реакцию на любое давление крови. 
Восстановление равновесного состояния получено в рамках его численной аппроксимации: конечно-разностная дискретизация (\ref{loc-force}) 
приводит к системе нелинейных алгебраических уравнений, которая должна быть решена итерационно методом Ньютона.

Численная модель <<волокно-пружина>> имеет преимущества прямые обобщения с другими типами волокон и, таким образом, 
может быть распространена на гораздо более широкий класс геометрий бляшек.

В численном эксперименте предполагается, что левая общая сонная артерия (№5 на Рис.\ref{ych}) повреждена протяженной атеросклеротической бляшкой 
с просветом 10\%, 30\%, 50\% и 100\%. Коэффициенты упругой бляшки взяты из~\cite{vassilevski:2011}. 
Профили скоростей в наружном сонном продолжении (№94) и артериях круга Виллиса (№104, 102) показаны на Рис.~\ref{sc}. 
Наиболее заметные изменения в скорости происходят в случае бляшек с просветом 30\% и 10\%. 
В малой артерии Виллисова круга (№104) и на продолжении левой наружной сонной артерии (№94) наблюдается значительное снижение скорости крови.


\subsection{Расчётная схема для моделирования течения в сосуде по одномерной модели}
Рассмотрим систему уравнений, описывающую кровоток в одиночном сосуде (\ref{eq:mass-conserv}) -- (\ref{eq:mom-conserv}) при условии
отсутствия источников/стоков ($\varphi=0$).
Из внешних сил, действующих на поток будем учитывать силу трения.
Для этого зададимся профилем скорости согласно \cite{smith:2002}:
$$
\tilde u(x, \xi) = u(x) \frac{\zeta + 2}{\zeta} \left[1 - \left(\frac{\xi}{r}\right)^\zeta\right],
$$
где $r$ -- радиус скругления, $\xi$ -- радиальная координата, $\zeta$ -- константа, определяющая профиль.
В результате интегрирования уравнений Навье-Стокса получим значение силы трения (см. \cite{boileau:2015}):

$$
\psi = -\frac{2 (\zeta + 2) \mu \pi u}{\rho A}.
$$

Запишем определяющую систему уравнений в виде

\begin{equation*}
    %\label{sys_of_eq}
    \begin{cases}
	&\dfrac{\partial A}{\partial t}+\dfrac{\partial Au}{\partial x}=0,\\[10pt]
	&\dfrac{\partial u}{\partial t}+u\dfrac{\partial u}{\partial x}+\dfrac{1}{\rho}\dfrac{\partial p}{\partial x}=\psi(u, A),\\[10pt]
	&p=\dfrac{4}{3}\sqrt{\pi}\dfrac{Eh}{A_0}(\sqrt{A}-\sqrt{A_0}).
    \end{cases}
\end{equation*}
Здесь использована замыкающая зависимость $p(A)$ из работы \cite{boileau:2015}, учитывающая
эластичные свойства стенок сосудов через параметры $E$ -- модуль упругости и $h$ -- толщина стенок сосуда.

Проведём обезразмеривание системы:
\begin{equation}
    \label{sys_of_eq1}
    \begin{cases}
	\dfrac{\partial A}{\partial t}+\dfrac{\partial Au}{\partial x}=0,\\[10pt]
	\dfrac{\partial u}{\partial t}+\dfrac{1}{2}\dfrac{\partial u^2}{\partial x} = -\dfrac{\partial p}{\partial x}-M_f \dfrac{u}{A},\\[10pt]
	p=M_p(\sqrt{A}-1).
    \end{cases}
    \end{equation}
В результате все физические параметры задачи определены через два безразмерных комплекса
$$
M_f=\frac{2(\zeta+2)\mu \pi L}{\rho A_0 U_0}, \quad
M_p=\frac{4\sqrt{\pi}Eh}{3 \rho U_0^2\sqrt{A_0}},
$$
где $L$ -- характерная длина сосуда, $A_0$ -- характерная площадь поперечного сечения сосуда, $U_0$ -- характерная скорость потока.
Первое из них описывает трение жидкости о стенки сосуда, второе -- эластичные свойства стенок сосуда.
Характерное время процесса при этом определится как $t_0 = L/U_0$.

Определяющая система дифференциальных уравнений (\ref{sys_of_eq1}) включает в себя два гиперболических уравнения.
Первое из них имеет вид уравнения переноса, второе -- уравнения Бюргерса.

{\bf Дискретизация по времени.}
Запишем входящие в систему (\ref{sys_of_eq1}) дифференциальные уравнения
в общем виде:
\begin{equation}
\label{eq:hyper}
\dfrac{\partial f}{\partial t}+\dfrac{\partial F(u, f)}{\partial x} = S(u, f).
\end{equation}
Для дискретизации по времени будем использовать явную схему c шагом $\tau$:
$$
\frac{\hat f - f}{\tau} + \frac{\partial F(u, f)}{\partial x} = S(u, f).
$$

{\bf Аппроксимация по пространству. TVD-схема.}
Известно, что пространственная аппроксимация гиперболического слагаемого схемой второго порядка точности неустойчива,
а использование схемы первого порядка (схемы против потока) приводит к большому влиянию численной диффузии на решение.
Для построение низкодиссипативной устройчивой схемы будем использовать метод TVD \cite{mazo1:2018},
которая заключается в комбинировании схем первого и второго порядка в зависимости от значения градиента искомой функции.

Запишем уравнение (\ref{eq:hyper}) в полудискретизованном с шагом $h$ виде в $i$-том узле расчётной сетки:
$$
\dfrac{\partial f_i}{\partial t}+\dfrac{F_{i + 1/2} - F_{i - 1/2}}{h} = S_i(u, f).
$$

Значение потока вычисленные по схеме первого и второго порядка точности определятся в виде
$$
\begin{aligned}
	&F^u_{i+1/2}=\begin{cases}
		F_i, &u_i>0\\
		F_{i+1},& u_i<0
	\end{cases}\\[10pt]
	&F^h_{i+1/2}=\dfrac{F_i + F_{i+1}}{2}.
\end{aligned}
$$

А само значение $F$ запишется в виде
$$
F_{i+1/2} = F^u_{i+1/2} + \phi(r) \left(F^h_{i+1/2} - F^u_{i+1/2}\right),
$$
где $\phi$ -- функция-ограничитель, а $r$ -- определяется через сеточный градиент искомой функции
$$
r = \frac{f_i - f_{i-1}}{f_{i+1} - f_{i}}.
$$
\clearpage
{\bf Тестовая задача о бегущей волне} -- классическая задача для тестирования схем аппроксимации гиперболических уравнений.
Постановка задачи имеет вид
\begin{equation*}
\dfrac{\partial f(x, t)}{\partial t}+\dfrac{\partial f(x, t)}{\partial x} = 0
\end{equation*}
и начальные условия
$$
f(x, 0) = \begin{cases}
	1, \quad \left|x\right| \leq 0.2,\\
	0, \quad \left|x\right| > 0.2.
\end{cases}
$$


Для тестирование схемы рассматривались ограничители вида 

Upwind: $ \phi(r)=0 $

MinMod: $
\begin{aligned}
	\phi(r)=\begin{cases}
	0, &r\leq 0\\
	r, &0<r\leq 1\\
	1, &r>1
    \end{cases}
\end{aligned}
$


MC: $
\begin{aligned}
     \phi(r)=\begin{cases}
	0, &r\leq0\\
	2r, &0<r\leq\frac{1}{3}\\
	\frac{1+r}{2}, &r\leq3\\
	2, &r>3
    \end{cases}
\end{aligned}
$ 


VanLeer:$
\begin{aligned}
    \phi(r)=\frac{r+|r|}{1+|r|}
\end{aligned}
$


SuperBee: $
\begin{aligned}
    \phi(r)=\begin{cases}
	0, &r\leq 0\\
	2r, &0<r\leq 0.5\\
	1, &0.5<r\leq 1\\
	r, &1<r\leq 2\\
	2, &r>2
    \end{cases}
\end{aligned}
$


Umist: $
\begin{aligned}
    \phi(r)=\begin{cases}
	0, &r\leq 0\\
	2r, &0<r\leq 0.2\\
	0.52+0.75r, &0.2<r\leq \frac{3}{7}\\
	0.75+0.25r, &\frac{3}{7}<r\leq \frac{7}{3}\\
	2, &r>\frac{7}{3}
    \end{cases}
\end{aligned}
$


Ospre: $
\begin{aligned}
   \phi(r)=
    \frac{1.5(r^2+r)}{r^2+r+1}
\end{aligned}
$


На основе описанного выше можно произвести расчёты и построить графики распределения объема по времени, а так же сравнить
заданные ограничители.
Задача решалась с шагом по пространству $h=1/30$ и шагом по времени $\tau=0.007$.

\begin{figure}[h!]
    \centering
    \includegraphics[width=0.6\linewidth]{03.jpeg}
    \caption{Распределение объёма в момент 0.3}
    \label{03}
\end{figure}

\begin{figure}[h!]
    \centering
     \includegraphics[width=0.6\linewidth]{07.jpeg}
    \caption{Распределение объёма в момент 0.7}
    \label{07}
\end{figure}


\begin{figure}[h!]
    \centering
     \includegraphics[width=1\linewidth]{norm.png}
    \caption{График погрешностей.}
    \label{norms}
\end{figure}

\begin{table}[h]
\centering
\caption {Значения погрешностей в момент 0.56 численного решения уравнения переноса}
\label{tab:norms_transport}
\begin{tabular}{|l|l|c|c|c|c|c|c|c|c|}
\hline
Ограничитель & Значение \\
\hline
SuperBee & 0.0659053\\
\hline
MC &  0.0659238\\
\hline
Umist & 0.0663728\\
\hline
Van Leer & 0.0685166\\
\hline
Ospre &  0.069493\\
\hline
MinMod & 0.0737164\\
\hline
Upwind  & 0.103783\\
\hline
\end{tabular}
\end{table}


На основе полученных данных можем сделать вывод, что для моделирования данной задачи не стоит использовать такие ограничители, как
Upwind и MinMod, а самыми точными оказались SuperBee и MC.
\clearpage
{\bf Тестовая задача: уравнение Бюргерса.} Рассмотрим уравнение вида
$$
\dfrac{\partial u(x, t)}{\partial t}+\dfrac12\dfrac{\partial u^2(x, t)}{\partial x} = 0
$$
с начальными условиями
$$
u(x, 0) = \begin{cases}
	1 - (x-1)^2, \quad \left|x-1\right| \leq 1,\\
	0, \quad \left|x-1\right| > 1.
\end{cases}
$$
Точное решением этой задачи на моменты времени $t<0.5$ будет иметь вид
$$
u_e = 1-\frac{{{\left( 1-\sqrt{1+4 t\, \left( t-x+1\right) }\right) }^{2}}}{4 {{t}^{2}}}
$$

Задача решалась с шагом по пространству  $h=0.1$ и шагом по времени $\tau=0.025$.
На рисунке \ref{fig:burgers} и в таблице \ref{tab:burgers} представлены решения и среднеквадратичные отклонения, полученные с использованием различных ограничителей.
Также как и в случае с задачей о бегущей волне, наиболее близкие к точному решению удалось получить с использованием ограничителя Superbee.

\begin{figure}[h]
\centering
\includegraphics[width=0.5\linewidth]{burgers.png}
\caption{Сравнение численного и точного решений уравнения Бюргерса на момент $t=0.5$}
\label{fig:burgers}
\end{figure}

\begin{table}[h]
\centering
\caption {Значения погрешностей в момент $t=0.5$ численного решения уравнения Бюргерса}
\label{tab:norms_burgers}
\begin{tabular}{|l|l|}
\hline
Ограничитель & Значение \\
\hline
SuperBee & 0.025357\\
\hline
Van Leer & 0.0279678\\
\hline
Upwind  & 0.050419\\
\hline
\end{tabular}
\end{table}

\clearpage
{\bf Тестовая задача: задача об одиночном импульсе.}
Рассмотрим задачу (\ref{sys_of_eq1}) о течении в канале с физическими параметрами, представленными в 
таблице \ref{tab:single_impulse_params}.

\begin{table}[h]
\centering
\caption {Параметры задачи об одиночном импульсе}
\label{tab:single_impulse_params}
\begin{tabular}{|l|l|}
\hline
$L$ & 2 м \\
\hline
$A_0$ & $\pi$ см$^2$\\
\hline
$h$ & $1.5$ мм\\
\hline
$\rho$  & 1050 кг/м$^3$\\
\hline
$\mu$  & 0 или 4 мПа$\cdot$с\\
\hline
$\zeta$  & 9\\
\hline
$E$  & 4$\cdot10^5$ Па\\
\hline
\end{tabular}
\end{table}

В качестве входного расхода используем функцию
$$
Q(t) = 10^{-6} \exp\left(-10^{-4}(t-0.05)^2\right) \quad m^3/s
$$
имеющую вид импульса с максимальным значением $10^{-6}$ м$^3$/с в момент времени $t=0.05$ сек.

В результате обезразмерирования получим следующие значения безразмерных комплексов, входящих в определяющую систему
$$
M_f = 0 \text{ или } 263.295,\quad  M_p = 7.5\cdot10^{6}.
$$

Задача решалась с шагом по пространству $h=5\cdot10^{-4}$ и шагом по времени $\tau=2 \cdot 10^{-5}$.
Численное решение на момент времени $t=10^{-3}$ представлено на рисунке \ref{fig:single_pulse_result}.

\begin{figure}[h]
\centering
\includegraphics[width=0.5\linewidth]{single_pulse.png}
\caption{Значение давления для задачи одиночного импульса на момент $t=10^{-3}$. Красная линия соответствует параметру $M_f=0$, синяя -- $M_f=263.295$}
\label{fig:single_pulse_result}
\end{figure}


\newpage
\printbibliography

\end{document}
