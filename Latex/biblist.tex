\addcontentsline{toc}{section}{Список используемой литературы}
\begin{thebibliography}{}
    \bibitem{litlink1}  Boileau E. et al A benchmark study of numerical schemes for one-dimensional arterial blood flow modelling // Int. J. Numer. Meth. Biomed. Engng. 2015.
    \bibitem{litlink2} F. J. Walburn, D. J. Schneck. A constitutive equation for whole human blood. Biorheology, 13 (1976), 201–210.
    \bibitem{litlink3} S. Kim, Y.I . Cho, A. H. Jeon, B. Hogenauer, K. R. Kensey. A new method for blood viscosity measurement. J. Non-Newtonian Fluid Mech., 94 (2000), 47-56.
    \bibitem{litlink4}  E. A. Evans, R. M. Hochmuth. Membrane viscoelasticity. Biophys. J., 16 (1976), no. 1, 111.
    \bibitem{litlink5} G. B. Thurston. Viscoelasticity of human blood. Biophys. J., 12 (1972), 1205–1217.
    \bibitem{litlink6} G.B. Thurston. Non-Newtonian viscosity of human blood: Flow induced changes in microstructure. Biorheology, 31 (1994), no. 2, 179–192.
    \bibitem{litlink7} P.C. F. Moller, J. Mewis, D. Bonn. Yield stress and thixotropy: on the difficulty of measuring yield stress in practice. Soft Matter, 2 (2006), 274–288.
     \bibitem{litlink8} P.C. F. Moller, J. Mewis, D. Bonn. Yield stress and thixotropy: on the difficulty of measuring yield stress in practice. Soft Matter, 2 (2006), 274–288.
    \bibitem{litlink9}  S.M. Hosseini, J.J. Feng. A particle-based model for the transport of erythrocytes in capillaries. Chem. Eng. Sci., 64 (2009), 4488-4497.
    \bibitem{litlink10} N. Bessonov, E. Babushkina, S.F. Golovashchenko, A. Tosenberger, F. Ataullakhanov, M. Panteleev, A. Tokarev, V. Volpert. Numerical modelling of cell distribution in blood flow. Math. Model. Nat. Phenom., 9 (2014), no. 6, 69-84.
    \bibitem{litlink11} A. Ya. Bunicheva, M. A. Menyailova, S. I. Mukhin, N. V. Sosnin, A. P. Favorskii. Studying the influence of gravitational overloads on the parameters of blood flow in vessels of greater circulation. Mathematical Models and Computer Simulations, 5 (2013), no. 1, 81-91.
    \bibitem{litlink12} A.Ya. Bunicheva, S.I. Mukhin, N.V. Sosnin, A.P. Favorskii. Numerical experiment in hemodynamics. Differential Equations, 40 (2004), no. 7, 984-999.
    \bibitem{litlink13} A.S. Kholodov. Some dynamical models of external breathing and haemodynamics accounting for their coupling and substance transport. Computer Models and Medicine Progress, Nauka, Moscow, 127-163, 2001.
    \bibitem{litlink14} Blausen.com staff. “Blausen gallery 2014”. Wikiversity Journal of Medicine. DOI:10.15347/wjm/2014.010
    \bibitem{litlink15} T.J. Pedley, X.Y. Luo. Modelling flow and oscillations in collapsible tubes. Theoretical and Computational Fluid Dynamics, 10 (1998), 277-294.
    \bibitem{litlink16} S.S. Grigorjan, Y.Z. Saakjan, A.K. Tsatutjan. To the theory of Korotkoff method. Biomechanics, (1984), no. 15-16,54-75.
    \bibitem{litlink17}J. Alastruey, K.H. Parker, J. Peiro, S.J. Sherwin. Lumped parameter outflow models for 1-D blood flow simulations: effect on pulse waves and parameter estimation. Communications in Computational Physics, 4 (2008), no. 2, 317-336.
    \bibitem{litlink18} Y. Vassilevski, S. Simakov, V. Salamatova, Y. Ivanov, T. Dobroserdova. Vessel wall models for simulation ofatherosclerotic vascular networks. Math. Model. Nat. Phenom., 6 (2011), no. 7, 82-99
    \bibitem{litlink19} https://app.rlsnet.ru/api/storage/books/jpg/253\_P1\_05\_02\_e
    
    \bibitem{litlink20} J.A.G. Rhodin  Architecture of the vessel wall / The Handbook of Physiology. The Cardiovascular System. Bethesda, Maryland. 1980. T. 2. С. 1–31.
    
\end{thebibliography}
